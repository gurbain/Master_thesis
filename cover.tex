% -*-latex-*-
% 
% For questions, comments, concerns or complaints:
% thesis@mit.edu
% 
%
% $Log: cover.tex,v $
% Revision 1.8  2008/05/13 15:02:15  jdreed
% Degree month is June, not May.  Added note about prevdegrees.
% Arthur Smith's title updated
%
% Revision 1.7  2001/02/08 18:53:16  boojum
% changed some \newpages to \cleardoublepages
%
% Revision 1.6  1999/10/21 14:49:31  boojum
% changed comment referring to documentstyle
%
% Revision 1.5  1999/10/21 14:39:04  boojum
% *** empty log message ***
%
% Revision 1.4  1997/04/18  17:54:10  othomas
% added page numbers on abstract and cover, and made 1 abstract
% page the default rather than 2.  (anne hunter tells me this
% is the new institute standard.)
%
% Revision 1.4  1997/04/18  17:54:10  othomas
% added page numbers on abstract and cover, and made 1 abstract
% page the default rather than 2.  (anne hunter tells me this
% is the new institute standard.)
%
% Revision 1.3  93/05/17  17:06:29  starflt
% Added acknowledgements section (suggested by tompalka)
% 
% Revision 1.2  92/04/22  13:13:13  epeisach
% Fixes for 1991 course 6 requirements
% Phrase "and to grant others the right to do so" has been added to 
% permission clause
% Second copy of abstract is not counted as separate pages so numbering works
% out
% 
% Revision 1.1  92/04/22  13:08:20  epeisach

% NOTE:
% These templates make an effort to conform to the MIT Thesis specifications,
% however the specifications can change.  We recommend that you verify the
% layout of your title page with your thesis advisor and/or the MIT 
% Libraries before printing your final copy.
\title{Calibration and Fusion of Stereoscopic Cameras and Optical Range Finder Sensors for Zero Gravity Targets Inspection}

\author{Gabriel P. Urbain}
% If you wish to list your previous degrees on the cover page, use the 
% previous degrees command:
%       \prevdegrees{A.A., Harvard University (1985)}
% You can use the \\ command to list multiple previous degrees
%       \prevdegrees{B.S., University of California (1978) \\
%                    S.M., Massachusetts Institute of Technology (1981)}
\department{Department of Electronics, Optronics and Signal Processing (DEOS)}

% If the thesis is for two degrees simultaneously, list them both
% separated by \and like this:
% \degree{Doctor of Philosophy \and Master of Science}
\degree{Master of Science in Aerospace Engineering}

% As of the 2007-08 academic year, valid degree months are September, 
% February, or June.  The default is June.
\degreemonth{October}
\degreeyear{2014}
\thesisdate{October 23, 2014}

%% By default, the thesis will be copyrighted to MIT.  If you need to copyright
%% the thesis to yourself, just specify the `vi' documentclass option.  If for
%% some reason you want to exactly specify the copyright notice text, you can
%% use the \copyrightnoticetext command.  
%\copyrightnoticetext{\copyright IBM, 1990.  Do not open till Xmas.}

% If there is more than one supervisor, use the \supervisor command
% once for each.
\supervisor{Alvar Saenz-Otero}{Principal Research Scientist, Space Systems Laboratory, MIT}
\supervisor{Daniel Alazard}{Professor, Department of Mathematics, Computer Science and Control (DMIA), ISAE Supaero}

% This is the department committee chairman, not the thesis committee
% chairman.  You should replace this with your Department's Committee
% Chairman.
\chairman{moi}{moi}
% Make the titlepage based on the above information.  If you need
% something special and can't use the standard form, you can specify
% the exact text of the titlepage yourself.  Put it in a titlepage
% environment and leave blank lines where you want vertical space.
% The spaces will be adjusted to fill the entire page.  The dotted
% lines for the signatures are made with the \signature command.
\maketitle

% The abstractpage environment sets up everything on the page except
% the text itself.  The title and other header material are put at the
% top of the page, and the supervisors are listed at the bottom.  A
% new page is begun both before and after.  Of course, an abstract may
% be more than one page itself.  If you need more control over the
% format of the page, you can use the abstract environment, which puts
% the word "Abstract" at the beginning and single spaces its text.

%% You can either \input (*not* \include) your abstract file, or you can put
%% the text of the abstract directly between the \begin{abstractpage} and
%% \end{abstractpage} commands.

% First copy: start a new page, and save the page number.
\cleardoublepage
% Uncomment the next line if you do NOT want a page number on your
% abstract and acknowledgments pages.
% \pagestyle{empty}
\setcounter{savepage}{\thepage}
\begin{abstractpage}
In many areas of robotics, vision is becoming more and more common in applications such as localization, automatic map construction, autonomous navigation, path following, inspection, monitoring or risky situation detection. With the increasing performances of embedded computers and the development of faster algorithms in the last few years, multi-sensors data fusion is considered as an opportunity to take better advantage of different sensors characteristics and stretch the limits. This project aims at implementing a multi-sensor data fusion algorithm involving two stereoscopic cameras of and a Time-of-Flight camera (\gls{ToF}) on in-space nano-satellites called SPHERES.\\
This document is the result of five months internship at the MIT SSL, USA as part of the final project of a double Master degree in Aerospace Engineering at ISAE, France and Electrical Engineering at UMONS, Belgium. The first chapter introduces the goal and the context of the project. The second chapter is dedicated to the theoretical aspect and aims at summarizing the required mathematical background and development. A third chapter analyzes concretely the implementation and finally, the results of two different experiments set will be detailed in the fourth chapter before to conclude.
\end{abstractpage}

% Additional copy: start a new page, and reset the page number.  This way,
% the second copy of the abstract is not counted as separate pages.
% Uncomment the next 6 lines if you need two copies of the abstract
% page.
% \setcounter{page}{\thesavepage}
% \begin{abstractpage}
% \input{abstract}
% \end{abstractpage}

\cleardoublepage

\section*{Acknowledgments}
I am grateful to all the great people who helped me throughout this internship, starting with Alvar Saenz-Otero, who welcomed me in his lab and always took the time to guide me, all SSL staff for their help and their advice, Daniel Alazard who accelerated the administrative process, Marina G. March who undertake this experience on my side as well as all my professors in Mons and Toulouse and especially Ir. S. Lizy-Destrez and Pr. T. Dutoit who helped me personally in all my projects.\\\\
I also would like to thank GDF Suez, the University of Mons and the Fernand Lazard Foundation for the financial support they provided during this internship.\\\\
Finally, I give the biggest thanks of all to my parents, my sister and my friends in Mons and Supaero who, through their support, their energy and their smile, made my six years at university the most unforgettable experience ever.

%%%%%%%%%%%%%%%%%%%%%%%%%%%%%%%%%%%%%%%%%%%%%%%%%%%%%%%%%%%%%%%%%%%%%%
% -*-latex-*-
